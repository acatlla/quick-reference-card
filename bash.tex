\documentclass[10pt,landscape]{article}
\usepackage{multicol}
\usepackage{calc}
\usepackage{ifthen}
\usepackage[landscape]{geometry}
\usepackage[utf8]{inputenc}
\usepackage[catalan]{babel}

% This sets page margins to .5 inch if using letter paper, and to 1cm
% if using A4 paper. (This probably isn't strictly necessary.)
% If using another size paper, use default 1cm margins.
\ifthenelse{\lengthtest { \paperwidth = 11in}}
	{ \geometry{top=.5in,left=.5in,right=.5in,bottom=.5in} }
	{\ifthenelse{ \lengthtest{ \paperwidth = 297mm}}
		{\geometry{top=1cm,left=1cm,right=1cm,bottom=1cm} }
		{\geometry{top=1cm,left=1cm,right=1cm,bottom=1cm} }
	}

% Turn off header and footer
\pagestyle{empty}
 

% Redefine section commands to use less space
\makeatletter
\renewcommand{\section}{\@startsection{section}{1}{0mm}%
                                {-1ex plus -.5ex minus -.2ex}%
                                {0.5ex plus .2ex}%x
                                {\normalfont\large\bfseries}}
\renewcommand{\subsection}{\@startsection{subsection}{2}{0mm}%
                                {-1explus -.5ex minus -.2ex}%
                                {0.5ex plus .2ex}%
                                {\normalfont\normalsize\bfseries}}
\renewcommand{\subsubsection}{\@startsection{subsubsection}{3}{0mm}%
                                {-1ex plus -.5ex minus -.2ex}%
                                {1ex plus .2ex}%
                                {\normalfont\small\bfseries}}
\makeatother


% Don't print section numbers
\setcounter{secnumdepth}{0}

\setlength{\parindent}{0pt}
\setlength{\parskip}{0pt plus 0.5ex}

% -----------------------------------------------------------------------

\begin{document}

\raggedright
\footnotesize
\begin{multicols}{3}


% multicol parameters
% These lengths are set only within the two main columns
%\setlength{\columnseprule}{0.25pt}
\setlength{\premulticols}{1pt}
\setlength{\postmulticols}{1pt}
\setlength{\multicolsep}{1pt}
\setlength{\columnsep}{2pt}

\begin{center}
     \Large{\textbf{Referència Bourne Again Shell (BASH)}} \\
\end{center}

\section{Patrons de coincidència}
\begin{tabular}{@{}ll@{}}
  \verb!?!                      & Qualsevol caràcter (únicament un)\\
  \verb!*!                      & Qualsevol cadena, fins i tot la cadena nula\\
  \verb![!\textit{abc}\verb!]!  & Qualsevols caràcter de dins els cortxets\\
  \verb![^!\textit{abc}\verb!]! & Qualsevols caràcter que no estigui dins els cortxets\\
\end{tabular}


\section{Variables d'entorn}
Les variables s'utilitzen amb el prefix \verb!$!\\
\begin{tabular}{@{}ll@{}}
  \verb!PATH! & Llista de directoris on buscar fitxers executables\\
  \verb!HOME! & Guarda la ruta del directori de conexió per l'usuari\\
  \verb!PWB! & Guarda la ruta del directori actual\\
\end{tabular}

\section{Cometes}
\begin{tabular}{@{}ll@{}}
  \verb!'! & No es realitza la substitució de nom var. per valor var\\
  \verb!"! & Substitució de nom var. per valor var\\
\end{tabular}

\section{Redirecció entrda/sortida}
\begin{tabular}{@{}ll@{}}
 \verb!>! & Redir. stdout destructiva\\
 \verb!>>! & Redir. stdout acomulativa\\
 \verb!<! & Redir. stdin\\
\end{tabular}

\section{Pippes}
Quan executem varis programes, en processos diferents a l'hora i volem que la sortida d'un estigui concatenada amb l'entrada de l'altre.\\
\begin{tabular}{@{}ll@{}}
  \verb!<proces1> | <proces2>! & Out \verb!proces1!, in \verb!proces2!\\
\end{tabular}

\section{Ajuda}
\begin{tabular}{@{}ll@{}}
 \verb!man <entrada>! & Pàgina manual de \verb!<entrada>!\\
 \verb!man <n> <entrada>! & Capítol \verb!<n>! del manual de \verb!<entrada>!\\
 \verb!info <entrda>! & Info. del sistema sobre \verb!<entrada>!\\
\end{tabular}

\section{Fitxers i deirectoris}
\begin{tabular}{@{}ll@{}}
  \verb!ls <d>! & Llista fitx. i dir. del dir. \verb!<d>!\\
  \verb!ls -l <d>! & Llista fitx. i dir. info. detallada\\
  \verb!ls -a <d>! & Llista fit. i dir. ocults i no ocults\\
  \verb!cd <d>! & Et mou al directori \verb!<d>!\\
  \verb!mkdir <d>! & Crea el directori \\
  \verb!cp <origen> <desti>! & Crea una copia\\
  \verb!rm <f>! & Elimina el fitxer\\
  \verb!rm -rf <d>! & Elimina dir. i el contingut\\
  \verb!touch <f>! & Crea fitx o modifica data modificació\\
  \verb!cat <f>! & Mostra per \textit{stdout} el fitxer\\
  \verb!convert <oldf> <nf>! & Converssió de mida, resolució i format\\
  \verb!chmod <opt> <f>! & Modifica els permisos del fitxer\\
\end{tabular}

\section{Busqueda d'arxius \texttt{find}}
La comanda \verb!find! busca al sistema arxius segons els criteris especificats a les opcions. La comanda es composa per \verb!find <dir> <condicios> <accions>! on \verb!<dir>! és el directori a partir del qual comença la cerca, \verb!<condicions>! són les condicions que a de complir el fitxer i \verb!<accions>! són les accions a aplicar als fitxers que trobem i compleixin les condicions de cerca.\\
\subsection{Expressions de valors numèrics}
\begin{tabular}{@{}ll@{}}
  \verb!+n! & Un num major que \verb!n!\\
  \verb!-n! & Un num menor que \verb!n!\\
  \verb!n! & Exactament el valor \verb!n!\\
\end{tabular}
\subsection{Opcions de la comanda \texttt{find}}
\begin{tabular}{@{}ll@{}}
  \verb!admin n! & S'ha accedit al fitxer fà \verb!n! minuts\\
  \verb!empty!  & El fitxer o directori està buit\\
  \verb!links n! & El fitxer té \verb!n! enllaços\\
  \verb!mmin n! & S'ha modificat el fitxer fà \verb!n! minuts\\
  \verb!name <patro>! & El nom coincideix amb el patrò\\
  \verb!size n! & El fitxer té la mida \verb!n!\\
  \verb!type c! & El fitxer és del tipus \verb!c!\\
\end{tabular}

\section{Filtres}
\begin{tabular}{@{}ll@{}}
  \verb!echo <c>! & Retorna la cadena \verb!<c>!\\
  \verb!read! & Llegeix de \textit{stdin}\\
  \verb!less! & Mostra info. pàg. a pàg.\\
  \verb!grep <patro>! & Retorna les línies que coincideixen \verb!patro!\\
  \verb!wc! & Conta el num. de char, word, line de l'entrda\\
  \verb!nl! & Numera les linies\\
  \verb!tr! & Remplaça o elimina un conjunt de caràcters\\
  \verb!head ! & Mostra les primeres línies\\
  \verb!tail ! & Mostra les últimes línies\\
  \verb!sort ! & Ordena les línies\\
  \verb!uniq! & Elimina les línies repetides(línies ordenades)\\
  \verb!cut! & Selecciona porcions d'una línia\\
  \verb!paste! & Uneix les línies de diferents fitxers\\
  \verb!tee! & Llegeix a \textit{stdin} i escriu a \textit{stdout} i un fitx.\\
  \verb!sq! & Compressió de llistes ordenades\\
  \verb!sed! & Editor de fluxos\\
\end{tabular}

\section{Tractament de text \texttt{awk}}
Comanda de manipulació de text. Comprova línia a línia si coincideix un partó. Si coincideix el patró, aplica accions.\\
\begin{tabular}{@{}ll@{}}
  \verb!awk 'BEGIN(<ini1>;<ini2>)!\\
  \verb!     <patro1> {<accio1>}!\\
  \verb!     <patro2> {<accio2>}!\\
  \verb!     END(<fin>; print "fi")' <fitxer>!\\
\end{tabular}


\section{Comunicació entre usuaris}
\begin{tabular}{@{}ll@{}}
  \verb!who! & Mostra quí està conectat al sistema\\
  \verb!whami! & Retorna id usuari que utilitzem\\
  \verb!write <user>! & Envia miss. a la terminal del usuari\\
  \verb!mesg! & Acceptar o no miss. d'altres usuaris\\
  \verb!wall <missatge>! & Envia \verb!missatge! a tots els usuaris\\
  \verb!talk <user@maquina>! & Inicia chat amb un usuari\\
  \verb!mail! & Llegir i enviar correu electrònic\\
\end{tabular}

\section{Altres utilitats}
\begin{tabular}{@{}ll@{}}
  \verb!cal! & Mostra el calendari del mes o any idicat\\
  \verb!date! & Mostra hora/data del sistema en diferents formats\\
  \verb!links! & Navegador web per consola\\
  \verb!! & \\
  \verb!! & \\
\end{tabular}

\section{Expressions regulars}
L'operador \verb!|! permet formar una expressió regular formada per dues expressions regulars on s'ha de complir una de les dues expressions que la formen. \verb!<exp reg1> | <exp reg2>!\\
\subsection{Classes de caràcters}
\begin{tabular}{@{}ll@{}}
  \verb![:alnum:]! & Caràcters alfanumèrics\\
  \verb![:alpha:]! & Caràcters alfabètics\\
  \verb![:cntrl:]! & Còdis de control\\
  \verb![:digit:]! & Dígits\\
  \verb![:graph:]! & Càracters gràfics\\
  \verb![:lower:]! & Lletres minúscula\\
  \verb![:print:]! & Caràcters imprimibles\\
  \verb![:punct:]! & Símbol de puntuació\\
  \verb![:space:]! & Espais i tabulacions\\
  \verb![:upper:]! & Lletra majúscula\\
  \verb![:xdigit:]! & Disgits headecimal\\
  \verb!.! & Qualsevol caràcter exepte el salt de línia \\
  \verb!^! & Representa l'inici de línia\\
  \verb!$! & Representa el final de línia\\
  \verb!\<! & Representa l'inici d'una paraula\\
  \verb!\>! & Representa el final d'una paraula\\
\end{tabular}
\subsection{Operadors d multiplicitat}
\begin{tabular}{@{}ll@{}}
  \verb!?! & L'elem. precedent és opcional\\
  \verb!*! & L'elem. precedent pot apareixer zero o més cops\\
  \verb!+! & L'elem. precedent apareix un o més cops\\
  \verb!{n}! & L'elem. precedent apareix n cops\\
  \verb!{n,}! & L'elem. precedent apareix com a mín n cops\\
  \verb!{,n}! & L'elem. precedent apareix com a màx n cops\\
  \verb!{n,m}! & L'elem. precedent apareix, mín n i màx m cops\\
\end{tabular}

\section{Shell scripts}

Cada programa que s'executa des de la shell guarda el codi de retorn a la variable \verb!$?!. Indica si el programa a finalitzat correctament (valor zero) o bé s'a produit algun error.

La comanda \texttt{zenity} permet crear finestres de diàleg.

\subsection{Arguments posicionals}
\begin{tabular}{@{}ll@{}}
  \verb!$0! & Nom del script\\
  \verb!$1! & Primer arg. posicional\\
  \verb!$2! & Segon arg. posicional\\
  \verb!$3! & Terçer arg. posicional\\
  \verb!$4! & Quart arg. posicional\\
  \verb!$5! & Cinqué arg. posicional\\
  \verb!$6! & Sisé arg. posicional\\
  \verb!$7! & Seté arg. posicional\\
  \verb!$8! & Vuité arg. posicional\\
  \verb!$9! & Nové arg. posicional\\
  \verb!$#! & Nombre d'arg.s\\
  \verb!$*! & Cadena amb tots els arg.\\
  \verb!shift! & Desplaça a l'esquerra els arg., \verb!$n! passa a ser \verb!$(n-1)!\\
\end{tabular}
\subsection{Operador \texttt{if} i operador \texttt{case}}
\begin{tabular}{@{}ll@{}}
  \verb!if <exp1>!       & \verb!case <variable> in!\\
  \verb!  then <ordre1>! & \verb!  <patro1>)!\\
  \verb!elif <exp2>!     & \verb!    <ordre1>;;!\\
  \verb!  then <ordre2>! & \verb!  <patro2>|<patro3>)!\\
  \verb!  else <ordre3>! & \verb!    <ordre2>;;!\\
  \verb!fi!              & \verb!esac!\\
\end{tabular}
\subsection{Comanda \texttt{test} i \texttt{[ ]}}
És equivalent \verb!test <arg>! i \verb![ <arg> ]!.Evalua una condició i retorna, \texttt{0} si la condició es compleix i \texttt{null} si la condició no es compleix.\\ 
Podem utilitzar els operadors lógics \texttt{!} (not), \texttt{-a} (and) i \texttt{-o} (or), per construïr una expressió lògica.\\
\begin{tabular}{@{}ll@{}}
  & Cert si ...\\
  \verb![ -f <f> ]! & El fitx. regular existeix\\
  \verb![ -x <f> ]! & El fitx. existeix i és executable\\
  \verb![ -r <f> ]! & El fitx. existeix i es pot llegir\\
  \verb![ -w <f> ]! & El fitx. existeix i es pot escriure\\
  \verb![ -d <f> ]! & És un directori\\
  \verb![ -s <f> ]! & El fitxer existeix i no està buit\\
  \verb![ <str> ]! & No és la cadena buida \\
  \verb![ -z <str> ]! & La longitud de la cadena és zero\\
  \verb![ -n <str> ]! & La longitud de la cadena no és zero\\
  \verb![ <str1>==<str2> ]! & Són iguals \\
  \verb|[ <str1>!=<str2> ]| & Són diferents \\
  \verb![ num1 -eq num2 ]! & Els numeros són iguals\\
  \verb![ num1 -ne num2 ]! & Els numeros són diferents\\
  \verb![ num1 -gt num2 ]! & num1$>$num2\\
  \verb![ num1 -ge num2 ]! & num1$<=$num2\\
  \verb![ num1 -lt num2 ]! & num1$<$num2\\
  \verb![ num1 -le num2 ]! & num1$<=$num2\\
\end{tabular}

\subsection{Operadors condicionals \texttt{\&\&} i \texttt{||}}
Operadors entre dues sentències, la primera sempre s'executa i la segona s'executa en funció del resultat de la primera.\\
L'operador \texttt{\&\&}, executa la primera sentència i només en el cas que aquesta sigui certa (retorna \verb!0!), executa la segona.\\
L'operador \texttt{||}, executa la primera sentència i només en el cas que aquesta sigui falsa (no retorna \verb!0!), executa la segona.\\

\subsection{Operador \texttt{while} i \texttt{for}}
\begin{tabular}{@{}ll@{}}
  \verb!while <exp>   ! &\verb!for <var> in <val1> <val2>!\\
  \verb!do!             &\verb!do!\\
  \verb!  <ordre>!      &\verb!  <ordre>!\\
  \verb!done!           &\verb!done!\\
\end{tabular}

Si substituïm \verb!while! per \verb!until!, el bucle s'executara sempre que \verb!<exp>! retorni fals.

\subsection{Funcions}
\begin{tabular}{@{}ll@{}}
  \verb!function <nom_funcio> () {! & \\
  \verb!  <accions>! & \\
  \verb!}! & \\
\end{tabular}

\section{Processos}
\begin{tabular}{@{}ll@{}}
  \verb!ps! & Mostra info. sobre els processos\\
  \verb!top! & Monitor de sistema\\
  \verb!uptime! & Mostra la càrega mitja del sistema\\
  \verb!kill! & Envia una senyal a un procés\\
  \verb!killall! & Envia una senyal a tots els processos\\
  \verb!xkill! & Envia una senyal a un procés, mode gràfic\\
  \verb!nice! & Assignar prioritat a un procés\\
  \verb!renice! & Modifica la prioritat d'un procés\\
  \verb!nohup! & Llença procés amb sortida desconectada del terminal\\
\end{tabular}\\

\rule{0.3\linewidth}{0.25pt}
\scriptsize

\newpage

\end{multicols}
\end{document}
